%%%%%%%%%%%%%%%%%%%%%%%%%%%%%%%%%%%%%%%%
%12pt: grandezza carattere
%a4paper: formato a4
%openright: apre i capitoli a destra
%twoside: serve per fare un documento fronteretro
%report: stile tesi (oppure book)
\documentclass[12pt,a4paper,openright,twoside]{report}

\usepackage[italian]{babel}
%\usepackage[latin1]{inputenc}
\usepackage[utf8]{inputenc}
\usepackage{fancyhdr}
\usepackage{indentfirst}
\usepackage{graphicx}
\usepackage{newlfont}
%librerie matematiche
\usepackage{amssymb}
\usepackage{amsmath}
\usepackage{latexsym}
\usepackage{amsthm}
\usepackage{listings}
\usepackage{chngcntr}
\usepackage[chapter]{minted}
%librerie per tabelle
\usepackage{tabularx}
\usepackage{threeparttable}

\oddsidemargin=30pt \evensidemargin=20pt%impostano i margini
\hyphenation{sil-la-ba-zio-ne pa-ren-te-si}

\usepackage[square, numbers, comma, sort&compress]{natbib} %Bibliografia
\usepackage{hyperref}
\usepackage{float}

\usepackage[pass]{geometry}

%comandi per l'impostazione della pagina, vedi il manuale della libreria fancyhdr per ulteriori delucidazioni
\pagestyle{fancy}\addtolength{\headwidth}{20pt}
\renewcommand{\chaptermark}[1]{\markboth{\thechapter.\ #1}{}}
\renewcommand{\sectionmark}[1]{\markright{\thesection \ #1}{}}
\rhead[\fancyplain{}{\bfseries\leftmark}]{\fancyplain{}{\bfseries\thepage}}
\cfoot{}

\linespread{1.3} %comando per impostare l'interlinea

\renewcommand{\listingscaption}{Codice}
\renewcommand{\listoflistingscaption}{Elenco dei Codici}

\begin{document}
%%%%%%%%%%%%%%%%%%%%%%%%%%%%%%%%%%%%%%%%
% FRONTESPIZIO
\newgeometry{hmarginratio=1:1}
\begin{titlepage}
\begin{center}
%\includegraphics[width=2.56in]{figures/logo/logo_unibo.png}\\
% {{\Large{\textsc{Alma Mater Studiorum $\cdot$ Universit\`a di
% Bologna}}}} \rule[0.1cm]{14.7cm}{0.1mm}
% \rule[0.5cm]{14.7cm}{0.6mm}
%\vspace{5mm}
{\small{\bf SCUOLA DI INGEGNERIA E ARCHITETTURA\\
\vspace{2mm}
Dipartimento di Informatica -- Scienza e Ingegneria\\
\vspace{2mm}
Corso di Laurea Magistrale in Ingegneria Informatica }}
%se Laurea Magistrale scrivere "Corso di Laurea Magistrale in Ingegneria Informatica"
\end{center}
%\vspace{11.9mm}

\vspace*{\fill}
\begin{center}
{\LARGE{\bf Simulazione di Fluidi in CUDA\\
\vspace{5mm}
}}
\end{center}
%\vspace{11.9mm}
\vspace*{\fill}
\par
\noindent
\vfill
\begin{minipage}[t]{0.47\textwidth}
{\normalsize{\bf Professori:\\
Prof. Dr. Stefano Mattoccia \\
Prof. Dr. Fabio Tosi

%\vspace{5mm}
%Correlatori:\\
%Dr. Antonio Iacobelli\\
}}
\end{minipage}
\hfill
\begin{minipage}[t]{0.47\textwidth}\raggedleft
{\normalsize{\bf Presentata da:\\
Enrico Minguzzi}}
\end{minipage}
\vspace{10mm} % da aumentare a 20 o 30 se non si inseriscono i correlatori
\begin{center}
{\normalsize{%\bf Sessione \\%inserire il numero della sessione in cui ci si laurea
Anno Accademico 2024-2025}}%inserire l'anno accademico a cui si è iscritti
\end{center}
\end{titlepage}

\restoregeometry


%%%%%%%%%%%%%%%%%%%%%%%%%%%%%%%%%%%%%%%%
% ABSTRACT 
\clearpage{\pagestyle{empty}\cleardoublepage}
\pagenumbering{roman}
\renewcommand{\abstractname}{Abstract}
\phantomsection
\addcontentsline{toc}{chapter}{Abstract}
\begin{abstract}
    Questo \`e l'abstract: un riassunto dell'introduzione di massimo 300 parole. Da scrivere alla fine.
\end{abstract}


%%%%%%%%%%%%%%%%%%%%%%%%%%%%%%%%%%%%%%%%
% RINGRAZIAMENTI 
\clearpage{\pagestyle{empty}\cleardoublepage}
\chapter*{Ringraziamenti}
\phantomsection
\addcontentsline{toc}{chapter}{Ringraziamenti}
Qui possiamo ringraziare il mondo intero!!!!!!!!!!\\
Ovviamente solo se uno vuole, non \`e obbligatorio.

\clearpage{\pagestyle{empty}\cleardoublepage}



%%%%%%%%%%%%%%%%%%%%%%%%%%%%%%%%%%%%%%%%
% INDICE 
\tableofcontents %crea l'indice
%imposta l'intestazione di pagina
\rhead[\fancyplain{}{\bfseries\leftmark}]{\fancyplain{}{\bfseries\thepage}}
\lhead[\fancyplain{}{\bfseries\thepage}]{\fancyplain{}{\bfseries
Indice}}



%%%%%%%%%%%%%%%%%%%%%%%%%%%%%%%%%%%%%%%%
% ELENCO DELLE FIGURE
\clearpage{\pagestyle{empty}\cleardoublepage}
\renewcommand{\listfigurename}{Elenco delle Figure}
\phantomsection
\addcontentsline{toc}{chapter}{Elenco delle Figure}
\listoffigures %crea l'elenco delle figure



%%%%%%%%%%%%%%%%%%%%%%%%%%%%%%%%%%%%%%%%
% ELENCO DELLE TABELLE
\clearpage{\pagestyle{empty}\cleardoublepage}
\renewcommand{\listtablename}{Elenco delle Tabelle}
\phantomsection
\addcontentsline{toc}{chapter}{Elenco delle Tabelle}
\listoftables %crea l'elenco delle tabelle


%%%%%%%%%%%%%%%%%%%%%%%%%%%%%%%%%%%%%%%%
% ELENCO DEI CODICI
\clearpage{\pagestyle{empty}\cleardoublepage}
\phantomsection
\addcontentsline{toc}{chapter}{Elenco dei Codici}
\listoflistings %crea l'elenco dei codici


%%%%%%%%%%%%%%%%%%%%%%%%%%%%%%%%%%%%%%%%
% INTRODUZIONE
\clearpage{\pagestyle{empty}\cleardoublepage}
\chapter{Introduzione} 
\lhead[\fancyplain{}{\bfseries\thepage}]{\fancyplain{}{\bfseries\rightmark}}
\pagenumbering{arabic} %mette i numeri arabi
Questa \`e l'introduzione: da scrivere alla fine, un breve riassunto su cosa si andr\`a ad affrontare nella tesi.
Le linee guida per la stesura della Tesi di Laurea sono al seguente link \href{https://ulis.se/thesis/}{https://ulis.se/thesis/}.

\section{Prima Sezione} %crea la sezione
Questa \`e la prima sezione.

Ora vediamo un elenco numerato: %crea un elenco numerato
\begin{enumerate}
\item primo oggetto
\item secondo oggetto
\item terzo oggetto
\item quarto oggetto
\end{enumerate}

%\begin{figure}[h] %crea l'ambiente figura; [h] sta per here, cioè la figura va qui
%\begin{center} %centra nel mezzo della pagina la figura \includegraphics[width=5cm]{figura.eps} inserisce una figura larga 5cm se si vuole usare va scommentata

%inserisce la legenda ed etichetta la figura con \label{fig:prima}
%\caption[legenda elenco figure]{legenda sotto la figura}\label{fig:prima}
%\end{center}
%\end{figure}


\section{Seconda Sezione}
Ora vediamo un elenco puntato:
\begin{itemize} %crea un elenco puntato
\item primo oggetto
\item secondo oggetto
\end{itemize}


\section{Altra Sezione}
Vediamo un elenco descrittivo:
\begin{description} %crea un elenco descrittivo
  \item[OGGETTO1] prima descrizione;
  \item[OGGETTO2] seconda descrizione;
  \item[OGGETTO3] terza descrizione.
\end{description}


\subsection{Altra SottoSezione}
%crea una sottosottosezione

\subsubsection{SottoSottoSezione}Questa sottosottosezione non viene
numerata, ma \`e solo scritta in grassetto.


\section{Altra Sezione} %crea una sottosezione
Vediamo la creazione di una tabella; la tabella \ref{tab:uno}
(richiamo il nome della tabella utilizzando la label che ho messo sotto):
la facciamo di tre righe e tre colonne, la prima colonna
``incolonnata'' a destra (r) e le altre centrate (c):\\
\begin{table}[h] %ambiente tabella (serve per avere la legenda)
\begin{center} %centra nella pagina la tabella
\begin{tabular}{r|c|c} %tre colonne con righe verticali prodotte con |
\hline \hline                           %inserisce due righe orizzontali
$(1,1)$ & $(1,2)$ & $(1,3)$\\ %& separa le colonne e con
\hline                                  %inserisce una riga orizzontale
$(2,1)$ & $(2,2)$ & $(2,3)$\\ %  \\ va a capo
\hline                                  %inserisce una riga orizzontale
$(3,1)$ & $(3,2)$ & $(3,3)$\\
\hline \hline                           %inserisce due righe orizzontali
\end{tabular}
\caption[legenda elenco tabelle]{legenda tabella}\label{tab:uno}
\end{center}
\end{table}


\section{Altra Sezione}\label{sec:prova}%posso mettere le label anche alle section
In questa sezione voglio fare un riferimento alla bibliografia: questo \`e il mio riferimento %\cite{gori2022metrics}.

\subsection{Listati dei programmi}
\subsubsection{Primo Listato}
\begin{verbatim}
        In questo ambiente     posso scrivere      come voglio,
lasciare gli spazi che voglio e non % commentare quando voglio
e ci sarà scritto tutto.
Quando lo uso è meglio che disattivi il Wrap del WinEdt
\end{verbatim}
\clearpage{\pagestyle{empty}\cleardoublepage}



%%%%%%%%%%%%%%%%%%%%%%%%%%%%%%%%%%%%%%%%
% SCENARI APPLICATIVI E STATO DELL'ARTE
%\chapter{Scenari applicativi e stato dell'arte}
%Cosa stiamo affrontando, p.e. una tesi che propone un attacco a HTTPS dovrebbe spiegare bene come funziona la parte di HTTPS attaccata e quali sono le limitazioni.
\chapter{Scenari applicativi e stato dell'arte}
Cosa stiamo affrontando, p.e. una tesi che propone un attacco a HTTPS dovrebbe spiegare bene come funziona la parte di HTTPS attaccata e quali sono le limitazioni.

\clearpage{\pagestyle{empty}\cleardoublepage}



%%%%%%%%%%%%%%%%%%%%%%%%%%%%%%%%%%%%%%%%
% ANALISI PROGETTUALE
\chapter{Analisi progettuale}
Cosa vogliamo raggiungere, perch\'e e quali sono state le vostre scelte per lo sviluppo ``su carta".


\clearpage{\pagestyle{empty}\cleardoublepage}



%%%%%%%%%%%%%%%%%%%%%%%%%%%%%%%%%%%%%%%%
% IMPLEMENTAZIONE
\chapter{Implementazione}
Come abbiamo implementato il programma, se ci sono state scelte particolari o punti particolarmente ostici che sono stati superati. Se ci sono state difficolt\`a insormontabili vanno spiegate.\\
\`E necessario inserire estratti di codice, tabelle o immagini nella tesi solamente quando questi aiutano la comprensione dell'argomento. Se pensate di stare scrivendo una parte molto tecnica e poco comprensibile al lettore, aiutatevi con le immagini o con il codice. Per inserire estratti di codice rifarsi all'esempio seguente oppure \`e possibile usare il package  \href{https://www.overleaf.com/learn/latex/Code\_listing}{listing}.
\begin{minted}{c}
#include <stdio.h>
int main() {
   // printf() displays the string inside quotation
   printf("Hello, World!");
   return 0;
}
\end{minted}


\clearpage{\pagestyle{empty}\cleardoublepage}



%%%%%%%%%%%%%%%%%%%%%%%%%%%%%%%%%%%%%%%%
% RISULTATI
\chapter{Risultati}
Descrivete e inserite qui gli esperimenti effettuati e i dati raccolti. Se avete grafici e dati, presentateli in questa sezione (nella forma in cui pensate sia pi\`u facile visualizzare quello che volete far trasparire dai dati). Scrivete sempre su che supporti hardware o software avete effettuato le prove e quante ne avete fatte (se rilevante).


\clearpage{\pagestyle{empty}\cleardoublepage}



%%%%%%%%%%%%%%%%%%%%%%%%%%%%%%%%%%%%%%%%
% CONCLUSIONI
\chapter{Conclusioni e sviluppi futuri}

Da scrivere alla fine, un breve riassunto di cosa si \`e affrontato, i risultati (se quanto indicato nell'introduzione come obiettivo \`e stato raggiunto) e come \`e possibile continuare la tesi (p.e. se qualcosa non \`e stato affrontato per motivi di tempo o limitazioni hardware).


\clearpage{\pagestyle{empty}\cleardoublepage}



%%%%%%%%%%%%%%%%%%%%%%%%%%%%%%%%%%%%%%%%%%
% RIMUOVERE LE APPENDICI SE NON UTILIZZATE
%imposta l'intestazione di pagina
\renewcommand{\chaptermark}[1]{\markright{\thechapter \ #1}{}}
\lhead[\fancyplain{}{\bfseries\thepage}]{\fancyplain{}{\bfseries\rightmark}}
\appendix %imposta le appendici
\chapter{Prima Appendice} %crea l'appendice
In questa Appendice non si \`e utilizzato il comando:\\
%\verb"" è equivalente all' ambiente verbatim,  ma si utilizza all'interno di un discorso.
\verb"\clearpage{\pagestyle{empty}\cleardoublepage}", ed infatti l'ultima pagina 8 ha l'intestazione con il numero di pagina in alto.
%imposta l'intestazione di pagina
\rhead[\fancyplain{}{\bfseries \thechapter \:Prima Appendice}]
{\fancyplain{}{\bfseries\thepage}}



\clearpage{\pagestyle{empty}\cleardoublepage}



\chapter{Seconda Appendice}
\rhead[\fancyplain{}{\bfseries \thechapter \:Seconda Appendice}]
{\fancyplain{}{\bfseries\thepage}}


\clearpage{\pagestyle{empty}\cleardoublepage}
%%%%%%%%%%%%%%%%%%%%%%%%%%%%%%%%%%%%%%%%%
% BIBLIOGRAFIA
\addcontentsline{toc}{chapter}{Bibliografia}
\label{Bibliography}
\bibliographystyle{IEEEtran}
\bibliography{src/bibliography}
\rhead[\fancyplain{}{\bfseries Bibliografia}]
{\fancyplain{}{\bfseries\thepage}}
\end{document}

